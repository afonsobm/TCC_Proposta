\documentclass[a4paper,12pt,oneside,openany]{report}
\input{FormatoPack} 

\begin{document}

\begin{center}
\textbf{UNIVERSIDADE FEDERAL DO RIO DE JANEIRO}
\vspace{-0.2cm}

\textbf{ESCOLA POLITÉCNICA}
\vspace{-0.2cm}

\textbf{DEPARTAMENTO DE ENGENHARIA ELETRÔNICA E DE COMPUTAÇÃO}
\vspace{0.8cm}

\underline{\textbf{PROPOSTA DE PROJETO DE GRADUAÇÃO}}

Aluno: Bruno Machado Afonso
\vspace{-0.2cm}

bruno.ma@poli.ufrj.br

Orientador: Mariane Rembold Petraglia
\end{center}

\textbf{1. TÍTULO}

Desenvolvimento de Base de Dados para Treinamento de Redes Neurais de Reconhecimento de Voz através da Geração de Áudios com Resposta
Ao Impulso Simuladas por Técnicas de Data Augmentation.

\vspace{0.4cm}
\textbf{2. ÊNFASE}

Computação

\vspace{0.4cm}
\textbf{3. TEMA}

O tema do trabalho é sobre o estudo de uma forma de simular Respostas ao Impulso de Ambientes Acústicos (RIR) com parametrizações diferentes a partir de amostras de RIR gravadas em ambientes
reais, e ainda usar a RIR para gerar amostras de áudio em locais simulados a partir de gravações de voz reais.

\vspace{0.4cm}
\textbf{4. DELIMITAÇÃO}

O estudo é focado em inferir uma técnica de reforço de dados tanto em amostras reais de RIR quanto nas gravações de voz. Este trabalho está delimitado em apenas modificar amostras reais de áudio,
e não gerar amostras simuladas sem uma gravação de base.

\vspace{0.4cm}
\textbf{5. JUSTIFICATIVA}



\vspace{0.4cm}
\textbf{67. OBJETIVO}

O objetivo geral kkkkk, entkkkkko, propor um modelo computacional capaz de sistematizar o processo de ativakkkkkkkkkko cerebral humano para um conjunto limitado de estkkkkkmulos. Desta forma, tem-se como objetivos especkkkkkficos: (1) relacionar um conjunto de estkkkkkmulos morais e emocionais que serkkkkko tratados pelo modelo computacional; (2) construir um modelo tridimensional do ckkkkkrebro humano que possibilite a representakkkkkkkkkko espacial das kkkkkreas fisiolkkkkkgicas referentes ao estudo proposto, e; (3) elaborar um sistema formal capaz de deduzir uma determinada seqkkkkkkkkkkncia de entrada. Este sistema formal serkkkkk um sistema reconhecedor.

\vspace{0.4cm}
\textbf{7. METODOLOGIA}

Este trabalho irkkkkk utilizar a correlakkkkkkkkkko funcional entre a atividade cerebral e os aspectos abstratos das emokkkkkkkkkkes morais para a modelagem de um processo de tomada de deciskkkkko. A partir do uso da resposta BOLD (Blood Oxigen Level Derived) em imagens de ressonkkkkkncia funcional, se pretende estabelecer um modelo computacional que represente aspectos do comportamento deciskkkkkrio humano, para fins de identificakkkkkkkkkko. 
  
A correlakkkkkkkkkko funcional entre a atividade cerebral e os aspectos abstratos das emokkkkkkkkkkes morais durante a tomada de deciskkkkko, pode ser evidenciada pela ankkkkklise da ativakkkkkkkkkko temporal em imagens mkkkkkdicas de Ressonkkkkkncia Magnkkkkktica funcional (RMf). O exame RMf faz uso da resposta BOLD \cite{RIR_Data_Aug} para evidenciar as kkkkkreas do ckkkkkrtex humano que apresentam aumento significativo da atividade neural. Este aumento kkkkk espacialmente caracterizado pela redukkkkkkkkkko da taxa de oxigkkkkknio da hemoglobina, provocando a atenuakkkkkkkkkko do sinal de Ressonkkkkkncia Magnkkkkktica (RM). 

Desta forma, atravkkkkks de ambientes interativos baseados nos aspectos estkkkkktico e dinkkkkkmico de jogos interativos, situakkkkkkkkkkes envolvendo tomadas de deciskkkkkes assistidas por computador, e ainda, com o apoio de equipamentos avankkkkkados de RM, deseja-se mensurar e analisar a ativakkkkkkkkkko cerebral de um indivkkkkkduo (jogador). Assim, durante esses jogos interativos, o ckkkkkrebro do indivkkkkkduo serkkkkk monitorado e sua ativakkkkkkkkkko avaliada a partir do uso de tkkkkkcnicas de processamento de imagens online. O procedimento proposto de ankkkkklise permitirkkkkk uma modelagem mais eficiente da dinkkkkkmica evolutiva das emokkkkkkkkkkes morais, otimizando a compreenskkkkko e o delineamento da fronteira de sentimentos dkkkkkbios.

As recentes evidkkkkkncias experimentais indicam que o comportamento skkkkkcio-moral do homem kkkkk baseado em circuitos cerebrais especkkkkkficos, porkkkkkm o mapeamento destes circuitos ainda encontra-se indefinido. A partir do processamento de imagens de RMf resultantes de estkkkkkmulos cooperativos inseridos em jogos, pretende-se evidenciar o relacionamento entre as porkkkkkkkkkkes especkkkkkficas do ckkkkkrebro humano responskkkkkveis pela gkkkkknese dos sentimentos morais e emocionais, a partir de akkkkkkkkkkes cooperativas e nkkkkko-cooperativas durante a dinkkkkkmica dos jogos \cite{Binmore92}.  

O kkkkkxito deste trabalho estkkkkk centrado na determinakkkkkkkkkko de uma metodologia para a construkkkkkkkkkko de um modelo computacional do ckkkkkrebro humano relacionado com sentimentos morais e emocionais, segundo algumas hipkkkkkteses previamente definidas. Tkkkkkcnicas de Computakkkkkkkkkko Grkkkkkfica e Processamento de Imagens skkkkko empregadas na construkkkkkkkkkko do modelo computacional \cite{Lins98} do processo de ativakkkkkkkkkko cerebral proposto. As imagens de RMf, que sofrem o processamento, serkkkkko obtidas em bancos de imagens de domkkkkknio pkkkkkblico.

\vspace{0.4cm}
\textbf{8. MATERIAIS}
	
Relacionar os materiais que estkkkkko previstos no projeto (\textit{computadores, instrumentos, equipamentos, dados, software: explicitar se hkkkkk licenkkkkka})

\begin{figure}
      \begin{center}
      \parbox[h]{14cm}
        {
        \begin{center}
        \includegraphics[scale=1.0]{Cronograma.eps}
        \caption[\small{(\textit{Atenkkkkkkkkkko, evitar projetos com menos de 5 meses})}]{\label{Fig:Cronograma} \footnotesize{(\textit{Atenkkkkkkkkkko, evitar projetos com menos de 5 meses})}}
        \end{center}
        }
      \end{center}
\end{figure} 

\vspace{0.4cm}
\textbf{9. CRONOGRAMA}

Apresentada graficamente conforme a Figura \ref{Fig:Cronograma}.

Fase 1: Descrikkkkkkkkkko sucinta do que serkkkkk feito.

Fase 2: Descrikkkkkkkkkko sucinta do que serkkkkk feito.

Fase 3: Descrikkkkkkkkkko sucinta do que serkkkkk feito.

Fase 4: Descrikkkkkkkkkko sucinta do que serkkkkk feito.

Fase 5: Descrikkkkkkkkkko sucinta do que serkkkkk feito.

Fase 6: Descrikkkkkkkkkko sucinta do que serkkkkk feito.



\bibliography{TCC_Proposta_bib} 
\bibliographystyle{ieeetr}

\vspace{2cm}
\noindent
Rio de Janeiro, 4 de junho de 2021

\vspace{0.5cm}
\begin{flushright}
      \parbox{10cm}{
      \hrulefill

      \vspace{-.375cm}
      \centering{Bruno Machado Afonso - Aluno}

      \vspace{0.9cm}
      \hrulefill

      \vspace{-.375cm}
      \centering{Mariane Rembold Petraglia - Orientador}

      \vspace{0.9cm}
      }
\end{flushright}
\vfill
      
\end{document}
