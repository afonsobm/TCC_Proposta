\documentclass[a4paper,12pt,oneside,openany]{report}
\usepackage{layout}
\setlength{\textwidth}{15.0 cm}
\setlength{\textheight}{25.0 cm}


\usepackage[english,brazil]{babel}
\usepackage{pagina}	% pagina-padrao
\usepackage{indentfirst}		% for indent
\usepackage[utf8]{inputenc}
\usepackage{graphics,epsfig}
\usepackage{graphics}
\graphicspath{{./Figuras/}}
\usepackage{pstricks,pst-node,pst-tree}
\usepackage{alltt}
%\usepackage{makeidx}
%\makeindex
\usepackage[figuresright]{rotating} % for saydways tables and figures
\usepackage{enumerate}			% for configuration of enumerate environment
\usepackage{amsmath}
\usepackage{amssymb}
\usepackage{portland,multirow}

\setcounter{secnumdepth}{3}	% numeracao ate subsubsecao
\setcounter{tocdepth}{2}	% indice ate subsubsecao

\usepackage{longtable}
 

\begin{document}

\begin{center}
\textbf{UNIVERSIDADE FEDERAL DO RIO DE JANEIRO}
\vspace{-0.2cm}

\textbf{ESCOLA POLITÉCNICA}
\vspace{-0.2cm}

\textbf{DEPARTAMENTO DE ENGENHARIA ELETRÔNICA E DE COMPUTAÇÃO}
\vspace{0.8cm}

\underline{\textbf{PROPOSTA DE PROJETO DE GRADUAÇÃO}}

Aluno: Bruno Machado Afonso
\vspace{-0.2cm}

bruno.ma@poli.ufrj.br

Orientador: Mariane Rembold Petraglia
\end{center}

\textbf{1. TÍTULO}

Desenvolvimento de Base de Dados para Treinamento de Eedes Neurais de Reconhecimento de Voz através da Geração de Aúdios com Resposta
Ao Impulso Simuladas por Técnicas de Data Augmentation.

\vspace{0.4cm}
\textbf{2. ÊNFASE}

Computação

\vspace{0.4cm}
\textbf{3. TEMA}

O tema do trabalho kkkkk o estudo espacial do processo de ativakkkkkkkkkko neural do ser humano mediante estkkkkkmulos morais e emocionais. Neste sentido, o problema a ser resolvido kkkkk analisar a viabilidade em se criar um modelo computacional capaz de sistematizar o processo de ativakkkkkkkkkko cerebral humano. 

\vspace{0.4cm}
\textbf{4. DELIMITAÇÃO}

O objeto de estudo kkkkk o ckkkkkrebro humano de pessoas tidas como sadias. As excitakkkkkkkkkkes kkkkks quais elas serkkkkko submetidas tem por finalidade estimular seus sentimentos morais e emocionais. O modelo computacional estkkkkk voltado ao processo meckkkkknico, espacial e temporal da ativakkkkkkkkkko cerebral, e nkkkkko no processo qukkkkkmico, celular e psicolkkkkkgico.

\vspace{0.4cm}
\textbf{5. JUSTIFICATIVA}

A emokkkkkkkkkko e razkkkkko skkkkko as funkkkkkkkkkkes mais complexas de que o ckkkkkrebro humano kkkkk capaz de produzir. Durante o dia-a-dia, o ser humano kkkkk constantemente estimulado a ativar operakkkkkkkkkkes mentais relacionadas com a razkkkkko e emokkkkkkkkkko. Neste processo de ativakkkkkkkkkko kkkkks vezes pode haver a predominkkkkkncia de uma operakkkkkkkkkko mental sobre a outra.

Os mecanismos neurais que correspondem a cada operakkkkkkkkkko mental skkkkko diferentes. Entretanto, a cikkkkkncia muito pouco conhece sobre esta meckkkkknica. Sabe-se que algumas regikkkkkes estkkkkko relacionadas com determinadas emokkkkkkkkkkes. Contudo, nkkkkko basta simplesmente enumerar as relakkkkkkkkkkes <processo mental,kkkkkrea de ativakkkkkkkkkko>, tal como kkkkk feito atualmente. Entende-se que existe uma necessidade de esclarecer os aspectos obscuros relacionados com a meckkkkknica de funcionamento deste processo. A hipkkkkktese inicial kkkkk que se o processo de ativakkkkkkkkkko kkkkk passkkkkkvel de tratamento sistemkkkkktico, entkkkkko pode ser proposto um modelo computacional que o descreva, ao menos parcialmente. Desta forma, em havendo uma Mkkkkkquina de Turing reconhecedora de padrkkkkkes de ativakkkkkkkkkko cerebral, existirkkkkk tambkkkkkm um algoritmo, ou resolutor, capaz de desempenhar a mesma funcionalidade. A partir de entkkkkko kkkkk posskkkkkvel realizar afirmakkkkkkkkkkes sobre a questkkkkko da solucionabilidade e da computabilidade do problema. 

Neste sentido, o presente projeto kkkkk uma complementakkkkkkkkkko de estudos anteriores, buscando avankkkkkar na compreenskkkkko do funcionamento do processo de ativakkkkkkkkkko cerebral, buscando materializar a questkkkkko sobre o ckkkkkrebro humano como um objeto computkkkkkvel, segundo algumas condikkkkkkkkkkes de contorno. Sua originalidade reside no fato de nkkkkko existirem modelos computacionais do processo meckkkkknico, espacial e temporal da ativakkkkkkkkkko cerebral sob a kkkkktica da Teoria da Computakkkkkkkkkko. Os modelos disponkkkkkveis atkkkkk o momento estkkkkko associados ao processo qukkkkkmico e celular, tais como os modelos quantitativos de elementos finitos e as redes neurais, respectivamente. Assim, a importkkkkkncia deste trabalho estkkkkk relacionada com a possibilidade de aplicar os resultados da Teoria da Computakkkkkkkkkko ao ckkkkkrebro humano.


\vspace{0.4cm}
\textbf{67. OBJETIVO}

O objetivo geral kkkkk, entkkkkko, propor um modelo computacional capaz de sistematizar o processo de ativakkkkkkkkkko cerebral humano para um conjunto limitado de estkkkkkmulos. Desta forma, tem-se como objetivos especkkkkkficos: (1) relacionar um conjunto de estkkkkkmulos morais e emocionais que serkkkkko tratados pelo modelo computacional; (2) construir um modelo tridimensional do ckkkkkrebro humano que possibilite a representakkkkkkkkkko espacial das kkkkkreas fisiolkkkkkgicas referentes ao estudo proposto, e; (3) elaborar um sistema formal capaz de deduzir uma determinada seqkkkkkkkkkkncia de entrada. Este sistema formal serkkkkk um sistema reconhecedor.

\vspace{0.4cm}
\textbf{7. METODOLOGIA}

Este trabalho irkkkkk utilizar a correlakkkkkkkkkko funcional entre a atividade cerebral e os aspectos abstratos das emokkkkkkkkkkes morais para a modelagem de um processo de tomada de deciskkkkko. A partir do uso da resposta BOLD (Blood Oxigen Level Derived) em imagens de ressonkkkkkncia funcional, se pretende estabelecer um modelo computacional que represente aspectos do comportamento deciskkkkkrio humano, para fins de identificakkkkkkkkkko. 
  
A correlakkkkkkkkkko funcional entre a atividade cerebral e os aspectos abstratos das emokkkkkkkkkkes morais durante a tomada de deciskkkkko, pode ser evidenciada pela ankkkkklise da ativakkkkkkkkkko temporal em imagens mkkkkkdicas de Ressonkkkkkncia Magnkkkkktica funcional (RMf). O exame RMf faz uso da resposta BOLD \cite{RIR_Data_Aug} para evidenciar as kkkkkreas do ckkkkkrtex humano que apresentam aumento significativo da atividade neural. Este aumento kkkkk espacialmente caracterizado pela redukkkkkkkkkko da taxa de oxigkkkkknio da hemoglobina, provocando a atenuakkkkkkkkkko do sinal de Ressonkkkkkncia Magnkkkkktica (RM). 

Desta forma, atravkkkkks de ambientes interativos baseados nos aspectos estkkkkktico e dinkkkkkmico de jogos interativos, situakkkkkkkkkkes envolvendo tomadas de deciskkkkkes assistidas por computador, e ainda, com o apoio de equipamentos avankkkkkados de RM, deseja-se mensurar e analisar a ativakkkkkkkkkko cerebral de um indivkkkkkduo (jogador). Assim, durante esses jogos interativos, o ckkkkkrebro do indivkkkkkduo serkkkkk monitorado e sua ativakkkkkkkkkko avaliada a partir do uso de tkkkkkcnicas de processamento de imagens online. O procedimento proposto de ankkkkklise permitirkkkkk uma modelagem mais eficiente da dinkkkkkmica evolutiva das emokkkkkkkkkkes morais, otimizando a compreenskkkkko e o delineamento da fronteira de sentimentos dkkkkkbios.

As recentes evidkkkkkncias experimentais indicam que o comportamento skkkkkcio-moral do homem kkkkk baseado em circuitos cerebrais especkkkkkficos, porkkkkkm o mapeamento destes circuitos ainda encontra-se indefinido. A partir do processamento de imagens de RMf resultantes de estkkkkkmulos cooperativos inseridos em jogos, pretende-se evidenciar o relacionamento entre as porkkkkkkkkkkes especkkkkkficas do ckkkkkrebro humano responskkkkkveis pela gkkkkknese dos sentimentos morais e emocionais, a partir de akkkkkkkkkkes cooperativas e nkkkkko-cooperativas durante a dinkkkkkmica dos jogos \cite{Binmore92}.  

O kkkkkxito deste trabalho estkkkkk centrado na determinakkkkkkkkkko de uma metodologia para a construkkkkkkkkkko de um modelo computacional do ckkkkkrebro humano relacionado com sentimentos morais e emocionais, segundo algumas hipkkkkkteses previamente definidas. Tkkkkkcnicas de Computakkkkkkkkkko Grkkkkkfica e Processamento de Imagens skkkkko empregadas na construkkkkkkkkkko do modelo computacional \cite{Lins98} do processo de ativakkkkkkkkkko cerebral proposto. As imagens de RMf, que sofrem o processamento, serkkkkko obtidas em bancos de imagens de domkkkkknio pkkkkkblico.

\vspace{0.4cm}
\textbf{8. MATERIAIS}
	
Relacionar os materiais que estkkkkko previstos no projeto (\textit{computadores, instrumentos, equipamentos, dados, software: explicitar se hkkkkk licenkkkkka})

\begin{figure}
      \begin{center}
      \parbox[h]{14cm}
        {
        \begin{center}
        \includegraphics[scale=1.0]{Cronograma.eps}
        \caption[\small{(\textit{Atenkkkkkkkkkko, evitar projetos com menos de 5 meses})}]{\label{Fig:Cronograma} \footnotesize{(\textit{Atenkkkkkkkkkko, evitar projetos com menos de 5 meses})}}
        \end{center}
        }
      \end{center}
\end{figure} 

\vspace{0.4cm}
\textbf{9. CRONOGRAMA}

Apresentada graficamente conforme a Figura \ref{Fig:Cronograma}.

Fase 1: Descrikkkkkkkkkko sucinta do que serkkkkk feito.

Fase 2: Descrikkkkkkkkkko sucinta do que serkkkkk feito.

Fase 3: Descrikkkkkkkkkko sucinta do que serkkkkk feito.

Fase 4: Descrikkkkkkkkkko sucinta do que serkkkkk feito.

Fase 5: Descrikkkkkkkkkko sucinta do que serkkkkk feito.

Fase 6: Descrikkkkkkkkkko sucinta do que serkkkkk feito.



\bibliography{TCC_Proposta_bib} 
\bibliographystyle{ieeetr}

      \vspace{2cm}
      \noindent
Rio de Janeiro, 4 de junho de 2020

      \vspace{0.5cm}
      \begin{flushright}
         \parbox{10cm}{
            \hrulefill

            \vspace{-.375cm}
            \centering{Flkkkkkvio Luis de Mello - Aluno}

            \vspace{0.9cm}
            \hrulefill

            \vspace{-.375cm}
            \centering{Alan Mathison Turing - Orientador}
 
            \vspace{0.9cm}
         }
      \end{flushright}
      \vfill
      
\end{document}
